\chapter{Aim}

The intention of this project is to produce animated and interactive plots on web pages.
These plots can convey more information and be more engaging than regular static graphics.
Currently, the creation of these plots is not possible in \R{} without encountering some difficulty.
A package for \R{} exists that can create the type of plots we want, but it is not capable of producing all but the most basic of statistical graphics.
This package, \gridSVG{}, needed to be improved upon in order to create useful animated and interactive graphics.
The \gridSVG{} package required extending so that it is able to generate plots that are sufficiently similar to those that are produced by \R{}.

\chapter{Introduction}

\section{What are web-based interactive graphics?}

It must be established what are web-based interactive graphics in order to illustrate the intended goal of this project.
Web-based graphics are images that are able to be viewed within a web page by a web browser.
The graphics we intend to produce are not only web-based, but also have the property of being animatable and interactive.
Interactivity involves changing the behaviour or appearance of an image, most commonly by the use of a mouse or keyboard.

\section{Existing solutions}

There are currently a few notable packages for \R{} that do allow for the creation of web-based interactive graphics.
It will be established how these packages work in order to explain why \gridSVG{} is being improved upon.

The \textsf{animation} package can create animated graphics in many formats, including GIF, Flash, and with the use of HTML and JavaScript any common image format can be animated.
In order to produce animated graphics, the \textsf{animation} package generates a series of static plots and pieces them together.
With enough static plots this gives the illusion of animation, much like how a film projector quickly shows a series of static frames when playing a movie.

\textsf{animation} relies heavily on the use of software not present within the package to produce many of the different graphics formats it supports.
In fact, the only formats that do not have any dependencies on third-party software are on-screen animations and HTML pages.
On-screen animations have the drawback of being unable to be stored in any way.
The GIF, Flash, PDF and video formats that \textsf{animation} all require software additional to \R{}.

Other packages have been released but they leverage other graphics systems to implement any animation or interactivity.
These packages include: \textsf{webvis}, which utilises the Protovis Javascript library, \textsf{googleVis}, which uses Google's Visualisation API, and \textsf{gWidgetsWWW} which provides its own Javascript library.
While these systems do all produce animated and interactive graphics, one problem is that they are no longer using \R{} graphics.
This means that a plot that is produced in \R{} will not appear similar to the resulting plots that these packages produce. 

A package that generates animated, interactive graphics via the \R{} graphics system is \textsf{SVGAnnotation} \citet{SVGAnnotation}.
It leverages \R{}'s \texttt{svg()} graphics device by post-processing the output to see which SVG elements correspond with components of a plot.
After performing the post-processing, animation can occur via SMIL and interactivity via JavaScript.

\section{Motivation for gridSVG}


\chapter{The design of gridSVG}

\section{What is grid? How does it work?}

\begin{center}
\begin{tikzpicture}[scale=3, >=triangle 45]
\tikzstyle{every node}=[draw,shape=circle];
\node[rounded rectangle, very thick] (grdev) at (1,1)  {\Large grDevices};
\node[rounded rectangle, very thick] (gr) at (0,2) {\Large graphics};
\node[rounded rectangle, very thick] (grid) at (2,2) {\Large grid};
\node[rounded rectangle, very thick] (svg) at (0,0) {\Large svg};
\node[rounded rectangle, very thick] (pdf) at (1,0) {\Large pdf};
\node[rounded rectangle, very thick] (gridsvg) at (2,0) {\Large gridSVG};
\draw [->, thick] (grdev) -- (svg);
\draw [->, thick] (grdev) -- (pdf);
\draw [->, thick] (gr) -- (grdev);
\draw [->, thick] (grid) -- (grdev);
\draw [->, thick] (grid) -- (gridsvg);
\end{tikzpicture}
\end{center}

\section{What is SVG? Javascript?}

\section{Mapping of grid grobs to SVG elements}
